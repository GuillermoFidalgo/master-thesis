
%Packages to include
\usepackage[utf8]{inputenc}
\usepackage{setspace}

\usepackage{mathpazo}
% \usepackage{times}

% \usepackage[letterpaper,margin=1in,
% width=150mm,top=30mm,bottom=1in,
% bindingoffset=6mm]{geometry} % for book
\usepackage[letterpaper,margin=1in]{geometry} % for article
\usepackage{graphicx,wrapfig,lipsum}
\usepackage{bm}
\usepackage{indentfirst}
\usepackage{verbatim}
\usepackage{float}
\usepackage[shortlabels]{enumitem}
\usepackage{subcaption}
\usepackage{titlesec}
\usepackage[font={footnotesize}]{caption}
\usepackage{amsmath}
\usepackage{amssymb}


% \usepackage[subfigure,titles]{tocloft}


\usepackage{url}
\urlstyle{same}
\usepackage{xfrac}
\usepackage[backend=biber,style=phys,
	articletitle=true,biblabel=brackets,%
	chaptertitle=false,pageranges=false%
]{biblatex}
\addbibresource{references.bib}
% \DeclareFieldFormat[report]{title}{\printtext[doi/url-link]{\mkbibemph{#1}}}
\usepackage[nottoc,numbib]{tocbibind}
% \settocbibname{References}

\usepackage[table,xcdraw]{xcolor}
\usepackage{siunitx}

\usepackage[%
	colorlinks=true,
	pdfborder={0 0 0},
	linkcolor=blue,
	citecolor=blue,
	urlcolor=blue
]{hyperref}

%\bibliography{references.bib}
\usepackage[nameinlink]{cleveref}

%\linenumbers

%Set images folder
\graphicspath{ {Images/} }

%Set Header and Footer for all pages
\usepackage{fancyhdr}

\pagestyle{fancy}
\fancyhead{}
\fancyhead[LO]{\nouppercase{\textbf{\leftmark}}}
\fancyhead[RO,LE]{\textbf{\thepage}}
\fancyhead[RE]{\nouppercase{\textbf{\rightmark}}}
\fancyfoot{}
\renewcommand{\chaptermark}[1]{\markboth{#1}{}}


% To have the transparent command
\usepackage{svg}


\setlist[itemize]{topsep=\parskip}

%\setlength\abovecaptionskip{-5ex}
%\setlength{\textfloatsep}{0pt plus 1.0pt minus 2.0pt}
\setlength\belowcaptionskip{-3ex}

\usepackage[labelfont=bf]{caption}

%Eliminate extra page after title and table of contents
\let\cleardoublepage=\clearpage

%Make signature and date lines for title page
\newcommand*{\SignatureAndDate}[1]{
	\par\noindent\makebox[3.0in]{\hrulefill} \hfill\makebox[2.0in]{\hrulefill}
	\par\noindent\makebox[2.5in][l]{#1}      \hfill\makebox[2.0in][l]{Date}
}

%Main Body

\usepackage{lineno}
\makeatletter
\def\makeLineNumberLeft{%
	\linenumberfont\llap{\hb@xt@\linenumberwidth{\LineNumber\hss}\hskip\linenumbersep}% left line number
	\hskip\columnwidth% skip over a column of text
	\rlap{\hskip\linenumbersep\hb@xt@\linenumberwidth{\hss\LineNumber}}\hss}% right line number
\leftlinenumbers% Re-issue [left] option
\makeatother
\linenumbers


% Adding custom packages
% \usepackage{physics}
