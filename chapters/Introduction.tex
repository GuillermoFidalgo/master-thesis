\chapter{Introduction}

The work for this thesis was performed with resources from European Organization for Nuclear Research (CERN)\footnote{\url{https://home.cern/about}}, the CMS Experiment\cite{CMS_detector}, and the LHC Physics Center (LPC) at Fermi National Lab (FNAL).
CERN was founded in 1954 and is located at the Franco-Swiss border near Geneva. At CERN, physicists and engineers are probing the fundamental structure of the universe. They use the world's largest and most complex scientific instruments to study the basic constituents of matter --- the fundamental particles.
The instruments used at CERN are purpose-built particle accelerators and detectors. Accelerators boost beams of particles to high energies before the beams are made to collide with each other or with stationary targets. Detectors observe and record the results of these collisions. The accelerator at CERN is called the Large Hadron Collider (LHC), the largest machine ever built by humans and it collides particles (mostly protons) at just
3 m/s under the speed of light.
The process gives physicists clues about how the particles interact and provides insights into the fundamental laws of nature. Nine\footnote{\url{https://home.cern/science/experiments}} experiments at the LHC use detectors to analyze particles produced by proton-proton collisions.
The biggest of these experiments, ATLAS and CMS, are general-purpose detectors designed to study the
fundamental nature of matter and fundamental forces and to look for new physics or evidence of particles that are beyond the Standard Model\footnote{\url{https://home.cern/about/physics/standard-model}}. Having two independently designed detectors is vital for cross-confirmation of any new discoveries made with minimal bias. The other two major detectors ALICE and LHCb, respectively, study a state of matter that was present just moments after the Big Bang and a preponderance of matter than antimatter.  Each experiment does important research that is key to understanding the universe that surrounds and makes us.

In particular, this work is focused on studies done for the Emerging Jets analysis and efforts on the development of tools that provide a mechanism to filter, evaluate and certify the quality of data collected in the CMS experiment.

% Add more on the origins of the QCD-like hidden sector. Talk about the need for DQM

\Cref{ch:CMS} presents a basic description of the Large Hadron Collider and the CMS Detector.
\Cref{ch:emj} presents a description and background on the Emerging Jets theory and analysis.
% \Cref{ch:HLT} develops the technical aspects and usage of the HLT system in CMS.
\Cref{ch:DQM} gives a brief description of what is Data Quality Monitoring (DQM) and its importance for CMS, as well as describe the Machine Learning tasks developed for it.
\Cref{ch:conclusion} summarizes the results of the analysis and ongoing DQM efforts.
