\chapter{Conclusion}\label{ch:conclusion}

There are three major topics of research that were discussed in this dissertation: The simulation studies involving the counting of L1-stubs for the HL-LHC CMS Inner Tracker upgrade ({stubs}), the overall 2016 search for SUSY in the all-hadronic channel using a customized top-tagger ({AnalysisChap}) and the improvements made for the estimation of the Z$\rightarrow \nu\bar{\nu}$+ jets background using an additional control region from $\gamma$+ jets events ({estimation}). These studies were explained in detail in their respective chapters and their individual results are provided. A summary of the most important results from each study is provided in this chapter.

\section{L1 Stub Counting for the HL-LHC CMS Tracker Upgrade}

Results from this study (detailed in {stubs}) reflect the overall effects that were expected beforehand. The removal of discs from the standard pixel geometry (consisting of 8 small and 4 large discs) results in a noticeable reduction of stub hits in the upgraded CMS Outer Tracker. This effect is specially apparent if the disc that is removed is closer to the interaction point, due to the much larger volume of particles that are present in this region. Therefore, the reduction in stubs is more pronounced when a small disc is removed (as in the case of the $7s4l$ geometry) than if a large disc is removed (as in the $8s3l$ pixel geometry). The reason for this effect stems from the fact that as particles travel through the various layers of the Inner Tracker material, some of them are bound to interact with it, producing particles that did not originate from the initial proton-proton collision. The stubs produced via such processes are considered to be ``fake'' stubs. To confirm these findings, an additional study was conducted using a sample that was virtually indistinguishable from the standard pixel geometry, but with the second disc on the positive side ``turned off'' or ``dead''. The results from this study confirm the initial findings and shows that there is indeed a correlation between the average number of stubs detected in the Outer Tracker and the total amount of material present in the upgraded Inner Tracker. An important factor that needs to be taken into account when interpreting these results is the re-optimization of the disc positions after removing a disc in the different pixel geometries considered. This feature could provide a possible explanation as to why the $6s3l$ geometry, which has two less small discs than the standard geometry (and one less large one), was found to have less of an effect on the average number of stubs than the $7s4l$ geometry.

\section{Search for SUSY in the All-Hadronic Channel}

The analysis presented in {AnalysisChap} shows the results of a search for SUSY in the 0-lepton final state using a customized top-tagger. The data was obtained from proton-proton collisions at the CMS detector during 2016 with a total integrated luminosity of 35.9 fb$^{-1}$ at a center-of-mass energy of 13 TeV. The search was conducted by specifying 84 non-overlapping regions of phase space with varying requirements on the $N_\text{b}$, $N_\text{t}$, $p_{\text{T}}^{miss}$, $H_\text{T}$ and $m_\text{T2}$ variables ({SearchBinDef}). Several dominant and non-dominant backgrounds were identified and estimated to account for all the majority of the processes that were seen in the collected data. The estimation procedures and their respective systematic and statistical uncertainties are discussed in {backgrounds}. The total background prediction vs. data for all 84 search bins ({SearchBinResults}) shows no statistically significant deviation from the predicted SM background. The biggest sources background were shown to be the t$\bar{\text{t}}$ and W+jets processes, followed by Z($\nu\bar{\nu}$)+jets, which were seen to be dominant in regions with a high $p_\text{T}$ threshold. Meanwhile, the contributions from the QCD multijet and rare backgrounds are found to be nearly negligible in all of the 84 search bins. Exclusion limits were calculated from these results for each of the signal models used, by applying a binned likelihood fit on the data. The likelihood function was obtained for each of the 84 search regions as well as for each of the background data control samples from the product of the Poisson probability density function. Exclusion limits were placed on the top squark, gluino and LSP production cross-sections with a 95\% confidence level (CL), calculated using a modified frequentist approach with the CL$_s$ criterion and asymptotic results for the test statistic. The 95\% CL exclusion limits obtained for the T2tt model, which consists of direct top squark production, excludes top squark masses up to 1020 GeV and LSP masses up to 430 GeV. For the T1tttt model, gluino masses of up to 2040 GeV and LSP masses up to 1150 GeV are excluded, with corresponding limits of 2020 and 1150 GeV for the T1ttbb model, 2020 and 1150 GeV for the T5tttt model, and 1810 and 1100 GeV for the T5ttcc model.
