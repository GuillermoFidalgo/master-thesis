\chapter{Abstract}

The Standard Model of Particle Physics (SM) has had a great track record over the decades. With the discovery of the top quark, the $\tau$ neutrino and the Higgs boson, the SM has proved it's effectiveness and prediction prowess. Yet, it leaves behind open questions regarding problems like Dark Matter and thus a need for new physics. This has brought up many exotic searches in hopes of answering the questions that the SM has yet to address.
To provide the necessary quality to search for new physics, physicists use the most complex machines ever designed.
The preponderance of cosmological evidence suggests that the density dark matter energy density of the Universe is around 5 times the amount of regular baryonic matter, and hence, experimental searches have been developed to explain this.
The CMS Collaboration has searched for signals of a dark matter model via the Emerging Jets analysis group.
As with all experiments in High Energy physics, acquiring high quality of data is paramount to achieve groundbreaking science. The CMS experiment achieves the collection of it's high quality data through the triggering and data acquisition systems put in place, but require manual labor to certify.
In this work I present trigger efficiency studies relevant to the Emerging Jets analysis. Moreover, I present my work
to improve the process of data certification in the DQM workflow implemented at the CMS Tracker DQM group. This work adds the automation of a new web application called the Machine Learning playground designed to improve DQM shifter efficiency in data certification.


% The Data Quality Monitoring (DQM) of CMS is a key asset to deliver high-quality data for physics analysis and it is used both in the online and offline environment. The current paradigm of the quality assessment is labor intensive and it is based on the scrutiny of a large number of histograms by detector experts comparing them with a reference. This project aims at applying recent progress in Machine Learning techniques to the automation of the DQM scrutiny. In particular the use of convolutional neural networks to spot problems in the acquired data is presented with particular attention to semi-supervised models (e.g. autoencoders) to define a classification strategy that doesn’t assume previous knowledge of failure modes. Real data from the hadron calorimeter of CMS are used to demonstrate the effectiveness of the proposed approach.

\vspace*{1cm}

\textit{Keywords}:  [Emerging Jets, Dark Matter, Quantum Chromodynamics, Machine Learning, Data Quality Monitoring]


\chapter{Resumen}

El Modelo Estándar de Física de Partículas (ME) ha tenido una historia existosa en las pasadas décadas. Con el descubrimiento del quark "cima", el neutrino $\tau$ y el bosón de Higgs, el ME ha mostrado su efectividad y su poder predictivo. Sin embargo, deja atrás preguntas abiertos con respecto a problemas como la Materia Oscura y por la tanto existe una necesidad de nueva física. Esto ha llevado a muchas búsquedas exóticas con la esperanza de responder a las preguntas que el Modelo Estándar aún no ha logrado responder. Para proporcionar la calidad necesaria para buscar nueva física, los físicos utilizan las máquinas más complejas que han sido diseñadas. La preponderancia de evidencia cosmológica sugiere que la densidad de energía de la materia oscura en el universo es aproximadamente 5 veces la densidad de materia bariónica regular. Por lo tanto se han desarrollado búsquedas experimentales para explicar esto. La Colaboración CMS ha buscado señales de un modelo de materia oscura a través del grupo de análisis de Jets Emergentes. Como en todos los experimentos en física de altas energías, adquirir datos de alta calidad es primordial para lograr ciencia innovadora.
El experimento CMS logra la recopilación de sus datos de alta calidad a través de los sistemas de ``trigger'' y adquisición de datos implementados pero requiere mucho trabajo manual para certificarlos. En este escrito, presento estudios de eficiencia de ``trigger'' relevantes para el análisis de Jets Emergentes. Además, presento mi trabajo para mejorar el proceso de la certificación de datos en el proceso de DQM implementado en el grupo de DQM del Tracker de CMS. Este trabajo añade la automatización de una nueva aplicación web llamada el ``Machine Learning Playground'', diseñada para mejorar la eficiencia de los trabajadores de turno de DQM en la certificación de datos.



\textit{Palabras claves}:  [Emerging Jets, Dark Matter, Quantum Chromodynamics, Machine Learning, Data Quality Monitoring]
