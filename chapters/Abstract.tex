\chapter*{Abstract}

The need of new physics has brought many exotic searches in hopes of answering the questions that the Standard Model has yet to address.
% The Data Quality Monitoring (DQM) of CMS is a key asset to deliver high-quality data for physics analysis and it is used both in the online and offline environment. The current paradigm of the quality assessment is labor intensive and it is based on the scrutiny of a large number of histograms by detector experts comparing them with a reference. This project aims at applying recent progress in Machine Learning techniques to the automation of the DQM scrutiny. In particular the use of convolutional neural networks to spot problems in the acquired data is presented with particular attention to semi-supervised models (e.g. autoencoders) to define a classification strategy that doesn’t assume previous knowledge of failure modes. Real data from the hadron calorimeter of CMS are used to demonstrate the effectiveness of the proposed approach. 

\vspace*{1cm}

\textit{Keywords}:  [Emerging Jets, Dark matter, Quantum Chromodynamics, Machine Learning, Data Quality Monitoring]


\chapter*{Resumen}
La necesidad de nueva física ha llevado a muchas búsquedas exóticas en esperanzas de contestar las preguntas que el Modelo Estándar de física de partículas no ha logrado responder.

% The Data Quality Monitoring (DQM) of CMS is a key asset to deliver high-quality data for physics analysis and it is used both in the online and offline environment. The current paradigm of the quality assessment is labor intensive and it is based on the scrutiny of a large number of histograms by detector experts comparing them with a reference. This project aims at applying recent progress in Machine Learning techniques to the automation of the DQM scrutiny. In particular the use of convolutional neural networks to spot problems in the acquired data is presented with particular attention to semi-supervised models (e.g. autoencoders) to define a classification strategy that doesn’t assume previous knowledge of failure modes. Real data from the hadron calorimeter of CMS are used to demonstrate the effectiveness of the proposed approach. 

\vspace*{1cm}

\textit{Palabras claves}:  [Emerging Jets, Dark matter, Quantum Chromodynamics, Machine Learning, Data Quality Monitoring]
