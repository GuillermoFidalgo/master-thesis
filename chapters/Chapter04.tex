\chapter{High-Level Trigger (HLT) \label{ch:HLT}}

With a beam spacing of \qty{25}{\ns}, beam crossings occur in the CMS detector at a rate of 40 million per second (\qty{40}{\MHz}). An additional complication is the approximately 25 interactions (referred to as `pileup', which is currently already much higher than the design value of 25) which occur with each beam crossing -- thus giving 1 billion events occurring in the CMS detector every second. In order to extract physics from these interactions it is vital to have fast electronics and very good resolution (proton-proton interactions are very messy and produce hundreds or thousands of particle candidates) and, because these events occur far too quickly to all be recorded and would take up vast amounts of disk space to store what are, for the majority, uninteresting events, very precise ``triggering'' is required.

Events of interest are selected using a two-tiered trigger system. The first level (L1), composed of custom hardware processors, uses information from the calorimeters and muon detectors to select events at a rate of around 100~\unit{kHz} within a fixed latency of \qty{4}{\us}~\cite{CMS:2020cmk}. The second level, known as the high-level trigger (HLT), consists of a farm of processors running a version of the full event reconstruction software optimized for fast processing, and reduces the event rate to around 1~\unit{kHz} before data storage~\cite{CMS:2016ngn}.


\section{Developing the Algorithm}
