\chapter{Emerging Jets (EJs) \label{ch:emj}}
%Section 3.1

% During data taking there are millions of collisions occurring in the center of the detector every second. The data per event is around one million bytes (1 MB), that is produced at a rate of about 
% 600 million events per second \cite{datataking}, that’s about 600 MB/s. Keeping in mind that only certain events are considered “interesting” for analysis, the task of deciding what events to consider out of all the data collected is a two-stage process. 
% First, the events are filtered down to 100 thousand events per second for digital reconstruction and then more specialized algorithms filter the data even more to around 100 $\sim$ 200 events per second that are found interesting.
% For CMS there is a Data Acquisition System that records the raw data to what’s called a High-Level Trigger farm which is a room full of servers that are dedicated to processing and classify this raw data quickly. 
% The data then gets sent to what’s known as the Tier-0 farm where the full processing and the first reconstruction of the data are done. \cite{cmscomputing} 

\section{Background information on EJs}

The Emerging Jets concept arises from the paper by P. Schwaller \cite{Schwaller:2015gea} where it was proposed to search for the Emerging Jets signature in the Run 1 dataset of the LHC Experiments to set limits on a combination of parameter ranges.


